\documentclass[10pt]{beamer}

\usetheme{metropolis}
\usepackage{appendixnumberbeamer}

\usepackage[spanish]{babel}
\usepackage[utf8]{inputenc}

\usepackage{booktabs}
\usepackage[scale=2]{ccicons}

\usepackage{pgfplots}
\usepgfplotslibrary{dateplot}

\usepackage{xspace}
\usepackage{epsfig}
\usepackage{color}
\newcommand{\themename}{\textbf{\textsc{metropolis}}\xspace}
\graphicspath{{./data/}{./resources/}}
\definecolor{mygray}{gray}{0.4}

\title{Taller de programación competitiva}
\subtitle{\textit{An amateur approach}}
\date{20 de febrero de 2018}
\author{
  Ignacio Ballesteros González\\
  {\color{mygray}\texttt{ballesteros@acm.org}\\}
  \\
  Samuel García Haro\\
  {\color{mygray}\texttt{samgh96@gmail.com}\\}
}

\institute{}
\titlegraphic{\hfill\includegraphics[height=1.5cm]{acm_png.png}}

\begin{document}

\maketitle

\begin{frame}{Índice}
  \setbeamertemplate{section in toc}[sections numbered]
  \tableofcontents[hideallsubsections]
\end{frame}

\section{Introduction}

\section{Estructuras de Datos básicas}

\section{Entrada/Salida}

\section{Métodos algorítmicos}

\subsection{Divide y vencerás}

\subsection{Método voraz}

\subsection{Programación dinámica}

\end{document}